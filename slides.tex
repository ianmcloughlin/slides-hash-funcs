\documentclass[dvipsnames,hidelinks,t]{beamer}

  % Enables the use of colour.
  \usepackage{xcolor}
  % Syntax high-lighting for code. Requires Python's pygments.
  \usepackage{minted}
  % Enables the use of umlauts and other accents.
  \usepackage[utf8]{inputenc}
  % Diagrams.
  \usepackage{tikz}
  % Settings for captions, such as sideways captions.
  \usepackage{caption}
  % Symbols for units, like degrees and ohms.
  \usepackage{gensymb}
  % Latin modern fonts - better looking than the defaults.
  \usepackage{lmodern}
  % Allows for columns spanning multiple rows in tables.
  \usepackage{multirow}
  % Better looking tables, including nicer borders.
  \usepackage{booktabs}
  % More math symbols.
  \usepackage{amssymb}
  % More math fonts, like mathbb.
  \usepackage{amsfonts}
  % More math layouts, equation arrays, etc.
  \usepackage{amsmath}
  % More theorem environments.
  \usepackage{amsthm}
  % More column formats for tables.
  \usepackage{array}
  % Adjust the sizes of box environments.
  \usepackage{adjustbox}
  % Better looking single quotes in verbatim and minted environments.
  \usepackage{upquote}
  % Better blank space decisions.
  \usepackage{xspace}
  % Better looking tikz trees.
  \usepackage{forest}
  % URLs.
  \usepackage{hyperref}
  % For plotting.
  \usepackage{pgfplots}
  
  % Various tikz libraries.
  % For drawing mind maps.
  \usetikzlibrary{mindmap}
  % For adding shadows.
  \usetikzlibrary{shadows}
  % Extra arrows tips.
  \usetikzlibrary{arrows.meta}
  % Old arrows.
  \usetikzlibrary{arrows}
  % Automata.
  \usetikzlibrary{automata}
  % For more positioning options.
  \usetikzlibrary{positioning}
  % Creating chains of nodes on a line.
  \usetikzlibrary{chains}
  % Fitting node to contain set of coordinates.
  \usetikzlibrary{fit}
  % Extra shapes for drawing.
  \usetikzlibrary{shapes}
  % For markings on paths.
  \usetikzlibrary{decorations.markings}
  % For advanced calculations.
  \usetikzlibrary{calc}
  % For layering.
  \usetikzlibrary{backgrounds}
  
  % GMIT colours.
  \definecolor{gmitblue}{RGB}{20,134,225}
  \definecolor{gmitred}{RGB}{220,20,60}
  \definecolor{gmitgrey}{RGB}{67,67,67}
  
  % Change some style options.
  \usetheme{metropolis}
  % Tell minted to use the following colour scheme. 
  \usemintedstyle{manni}
  % Remove some of the vertical space after the title.
  % \addtobeamertemplate{frametitle}{}{\vspace{-3mm}}
  % Change the default theme colours.
  \setbeamercolor{normal text}{fg=darkgray, bg=white}
  \setbeamercolor{alerted text}{fg=gmitred, bg=white}
  \setbeamercolor{example text}{fg=gmitblue, bg=white}
  \setbeamercolor{frametitle}{fg=gmitblue, bg=white}
  \setbeamercolor*{item}{fg=gmitblue}
  % Use a better math mode font.
  \usefonttheme[onlymath]{serif}
  % Don't display section pages.
  \metroset{sectionpage=none}
  % Change the default itemize bullets.
  \setbeamertemplate{itemize item}{\color{gray}--}
  % Change the position of left aligned math.
  %\setlength{\mathindent}{7mm}

  % An environment for displaying math in red, without lots of vertical space.
  \newcommand{\redmath}[1]{\vspace{-3mm} {\begin{center} \color{gmitred} $ #1 $ \end{center}} \vspace{-2mm}}

  % For displaying a blank character.
  \newcommand{\bl}{\underline{\hspace{2mm}}}

  % \citeurl can be used to a clickable short url to a slide as a reference.
  \renewcommand\footnoterule{}
  \newcommand{\citeurl}[1]{\let\thefootnote\relax\footnotetext{\tiny \textcolor{gmitgrey}{\href{http://#1}{#1}}}}
  \newcommand{\citeeg}[1]{\let\thefootnote\relax\footnotetext{\tiny \textcolor{gmitgrey}{#1}}}
  
  % Prevent minted from showing errors.
  \makeatletter
  \expandafter\def\csname PYGdefault@tok@err\endcsname{\def\PYGdefault@bc##1{{\strut ##1}}}
  \makeatother
  
  \begin{document}
    \title{Hash functions}
    \subtitle{}
    \author{ian.mcloughlin@gmit.ie}
    \date{}
  
    \begin{frame}
      \titlepage
    \end{frame}
  
    \begin{frame}{Binary functions}
  \redmath{f:\{0,1\}^* \rightarrow \{0,1\}^*}

  \begin{itemize}
    \item $f$ is a function that converts strings over $\{0,1\}$ into other strings over $\{0,1\}$.
    \item $f$ can be viewed as a map, a subset of $\{0,1\}^* \times \{0,1\}^*$ where no two first elements are equal and every element in $\{0,1\}^*$ is a first element.
    \item For example, $(0110,11111101) \in \{0,1\}^* \times \{0,1\}^*$.
  \end{itemize}
\end{frame}


\begin{frame}{Hash functions}
  \redmath{f:\{0,1\}^* \rightarrow \{0,1\}^n}

  \begin{itemize}
    \item A (binary) hash function is a function where the output is of fixed size.
    \item In the example above, $\{0,1\}^n$ has size $2^n$.
    \item Any set of fixed size will do, but we usually use $\{0,1\}^n$ for some $n \in \mathbb{N}$.
    \item For example, we could use $\{0, 1, 00, 01, 10, 11\}$ instead.
  \end{itemize}
\end{frame}

\begin{frame}{Using Hexadecimal}
  The benefit of using hex is that it saves space while being easy to convert to.
  Every nibble becomes a hex symbol.
  \begin{table}
    \centering
    \begin{tabular}{rl@{\hspace*{8mm}}rl@{\hspace*{8mm}}rl@{\hspace*{8mm}}rl}
      \toprule
      0 & 0000 & 1 & 0001 & 2 & 0010 & 3 & 0011 \\
      4 & 0100 & 5 & 0101 & 6 & 0110 & 7 & 0111 \\
      8 & 1000 & 9 & 1001 & A & 1010 & B & 1011 \\
      C & 1100 & D & 1101 & E & 1110 & F & 1111 \\
      \bottomrule
    \end{tabular}
  \end{table}

  \begin{exampleblock}{Example}
    \begin{table}
      \centering
      \begin{tabular}{cccccccc}
        \toprule
          0100 & 0111 & 0100 & 1101 & 0100 & 1001 & 0101 & 0100 \\
          4    & 7    & 4    & D    & 4    & 9    & 5    & 4     \\
        \bottomrule
      \end{tabular}
    \end{table}
    Try 010010010110000101101110 in your own time.
  \end{exampleblock}
\end{frame}

\begin{frame}{Common hash functions}
  \begin{exampleblock}{Input string}
    {\footnotesize Galway-Mayo Institute of Technology, Dublin Road, Galway, H91 T8NW}
  \end{exampleblock}
  \begin{exampleblock}{Outputs}
    \begin{table}
      \centering
      \begin{tabular}{rl}
        \toprule
          MD5    & {\tiny 4EC47B38AE21FD11A5E4993995097861} \\
          SHA1   & {\tiny 67481B1FF7E145067C12B7C6C5E681CD7EFDEDD6} \\
          SHA256 & {\tiny BE9863B550CC931D220E7EB08B7AFE44B70ED467A5015F34ED9DECA1B84F7A2D} \\
          CRC32  & {\tiny 22} \\
        \bottomrule
      \end{tabular}
    \end{table}
  \end{exampleblock}
  \begin{exampleblock}{Try yourself}
    Try ``Colorless green ideas sleep furiously'' and the empty string.
  \end{exampleblock}
\end{frame}

\begin{frame}{Properties}
  \begin{description}
    \setlength\itemsep{6mm}
    \item[Uniform:] outputs should have the same probability as each other.
    \item[Prefect:] every input gives a different output.
    \item[Minimal:] outputs form a coninuous range of bit strings.
    \item[Cryptographic:] difficult to calculate the input from the output.
    \item[Deterministic:] always the same output for a given input.
  \end{description}
\end{frame}

\begin{frame}{Loop-up tables}
  \begin{exampleblock}{Exercise}
    Write an algorithm that counts the number of bits set in an integer.
  \end{exampleblock}
\end{frame} 
  \end{document}
